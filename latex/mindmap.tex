\documentclass[oneside,a4paper]{article}
\usepackage [spanish] {babel} 
\usepackage [T1]{fontenc}
\usepackage [utf8]{inputenc}
\usepackage{tikz}
\usetikzlibrary{mindmap,trees}
\usepackage{geometry}
\geometry{a4paper, margin=2.5cm}

\begin{document}

\section*{Mindmap con Tikz}

Este mindmap lo usé para un apunte de una materia, está bueno como queda.

\begin{figure}[h!]
	\centering
	\begin{tikzpicture}
	\path[
	mindmap, concept color=white, text=black, 
	% Distancia y angulos de los hijos
	level 1/.append style={level distance=3.5cm, sibling angle=36},
	% Distancia y angulo de los nietos 
	level 2/.append style={level distance=2.5cm, sibling angle=30}, 
	% Escala del gráfico completo
	every node/.append style={scale=0.7}
	]
	
	% Nodo padre
	node[concept] {\textbf{Procesamiento digital de señales}}
	% De donde empezar el primer hijo, en grados respecto a la horizontal
	[clockwise from=0]
	% Cada hijo sigue el mismo formato
	child[concept color=blue!30] { 
		node[concept] {\textbf{Otras}} 
		[clockwise from=30]
		child { node[concept] {Redes de datos} }
		child { node[concept] {Adquisición de datos} }
		child { node[concept] {Simulaciones} }
	}
	child[concept color=blue!30] {
		node[concept] {\textbf{Medicina}}
		[clockwise from=-21]
		child { node[concept] {Diagnóstico por imagen} }
		child { node[concept] {Señales (ECG, EEG)} }
	}
	child[concept color=blue!30] { 
		node[concept] {\textbf{Telefonía}} 
		[clockwise from=-42]
		child { node[concept] {Reducción de ruido} }
		child { node[concept] {Reducción de eco} }
		child { node[concept] {Filtrado} }
	}
	child[concept color=blue!30] { 
		node[concept] {\textbf{Aplicaciones militares}} 
		[clockwise from=-93]
		child { node[concept] {Radares y sonares} }
		child { node[concept] {Criptografía} }
	} 
	child[concept color=blue!30] { 
		node[concept] {\textbf{Entreteni\-miento}} 
		[clockwise from=-106]
		child { node[concept] {Compresión de Audio} }
		child { node[concept] {Efectos especiales} }
		child { node[concept] {\textit{Streaming} de video} }
	}
	child[concept color=blue!30] {
		node[concept] {\textbf{Industria Aeroespacial}}
		[clockwise from=-150]
		child { node[concept] {Compresión de datos} }
		child { node[concept] {Mejoras de imágenes espaciales} }
		child { node[concept] {Analisis de datos de sensores} }
	}
	;
	\end{tikzpicture}
	\caption{Algunos campos de aplicación del proceso digital de señales.}
	\label{fig:aplicaciones-dsp}
\end{figure}

\end{document}